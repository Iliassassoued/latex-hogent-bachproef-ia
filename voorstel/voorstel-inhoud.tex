%---------- Inleiding ---------------------------------------------------------


\section{Inleiding}%
\label{sec:inleiding}

%skelet: 
%1. korte uitleg van probleem (information overload/Enterprise search issues) 
% -> bron zoeken die dit verder aantoont
%2. beschrijving van de issue in de context van Ziggu
%3. Ai als oplossing 
% -> bron zoeken die ai als oplossing bied voor de issues 
%3. onderzoeksvraag 
%  -> segue naar de onderzoeksvraag
%4. Wat we willen bereiken
%  -> PoC 
%5. Wat we hiervoor gaan opzoeken 

Het bedrijf Ziggu BV, een Belgisch Software-as-a-Service (SaaS) bedijf, biedt klantenportalen aan projectteams over heel deel verschillende sectoren. Eén van die sectoren die enorm gebruik maakt van het platform en die ook bekend staat voor zijn complexiteit \autocite{MohdNawi2014}, is de bouw & vastgoedsector . Deze complexiteit komt enerzijds door de vele betrokken partijen bij een bouwproject (denk: aannemer, projectontwikkelaar, klanten, leads,etc \autocite{MohdNawi2014}), maar anderzijds ook door de lange duurtijd met verschillende fases die vaak asynchroon lopen tussen verschillende units of projecten over verschillende fases heen. Hierdoor is er een grote hoeveelheid aan informatie voor hande die vaak moeilijk te doorzoeken is. 

Een concreet voorbeeld uit de praktijk in Ziggu is als volgt: Een projectontwikkelaarbedrijf is bezig met de bouw van een nieuw appartementencomplex. Dit complex zal een capaciteit hebben van ongeveer 100 units die kunnen worden verkocht aan mensen die op zoek zijn naar een nieuwe woning. Een project zoals deze loopt over een lange periode en zijn er per unit verschillende fases die doorlopen moeten worden. Denk hierbij aan een verkoopfases waar mogelijkse leads gezocht worden, een bouwfase waar klanten vaak keuzes moeten maken omtrent hun toekomstige woning zoals de keuken, badkamer,etc, een nazorg fase waarin defecten of problemen kunnen gemeld, etc. Elk van deze fases zal door iedere unit moeten doorlopen worden. De kans dat units in verschillende fases zitten is dus enorm groot. Als een klantenbegeleider van het project dan op zoek is naar informatie van een bepaalde unit zoals bijvoorbeeld welke persoon had een probleem met zijn keuken, dan is het vaak moeilijk om deze informatie eenvoudig en snel terug te vinden. 

Dit probleem van informatie terugvinden is zeker niet uniek voor Ziggu of de bouwsector. Zoals \textcite{Hawking2004} aankaart is dit probleem iets dat vanaf de rijs van het Web reeds bestaat. Hoewel hij in zijn onderzoek door het gebrek aan geavanceerde technologiëen in de tijd van de paper nog niet meteen aan een oplossing kon komen,  




%---------- Stand van zaken ---------------------------------------------------

\section{Literatuurstudie}%
\label{sec:literatuurstudie}

Hier beschrijf je de \emph{state-of-the-art} rondom je gekozen onderzoeksdomein, d.w.z.\ een inleidende, doorlopende tekst over het onderzoeksdomein van je bachelorproef. Je steunt daarbij heel sterk op de professionele \emph{vakliteratuur}, en niet zozeer op populariserende teksten voor een breed publiek. Wat is de huidige stand van zaken in dit domein, en wat zijn nog eventuele open vragen (die misschien de aanleiding waren tot je onderzoeksvraag!)?

Je mag de titel van deze sectie ook aanpassen (literatuurstudie, stand van zaken, enz.). Zijn er al gelijkaardige onderzoeken gevoerd? Wat concluderen ze? Wat is het verschil met jouw onderzoek?

Verwijs bij elke introductie van een term of bewering over het domein naar de vakliteratuur, bijvoorbeeld~\autocite{Hykes2013}! Denk zeker goed na welke werken je refereert en waarom.

Draag zorg voor correcte literatuurverwijzingen! Een bronvermelding hoort thuis \emph{binnen} de zin waar je je op die bron baseert, dus niet er buiten! Maak meteen een verwijzing als je gebruik maakt van een bron. Doe dit dus \emph{niet} aan het einde van een lange paragraaf. Baseer nooit teveel aansluitende tekst op eenzelfde bron.

Als je informatie over bronnen verzamelt in JabRef, zorg er dan voor dat alle nodige info aanwezig is om de bron terug te vinden (zoals uitvoerig besproken in de lessen Research Methods).

% Voor literatuurverwijzingen zijn er twee belangrijke commando's:
% \autocite{KEY} => (Auteur, jaartal) Gebruik dit als de naam van de auteur
%   geen onderdeel is van de zin.
% \textcite{KEY} => Auteur (jaartal)  Gebruik dit als de auteursnaam wel een
%   functie heeft in de zin (bv. ``Uit onderzoek door Doll & Hill (1954) bleek
%   ...'')

Je mag deze sectie nog verder onderverdelen in subsecties als dit de structuur van de tekst kan verduidelijken.

%---------- Methodologie ------------------------------------------------------
\section{Methodologie}%
\label{sec:methodologie}

Hier beschrijf je hoe je van plan bent het onderzoek te voeren. Welke onderzoekstechniek ga je toepassen om elk van je onderzoeksvragen te beantwoorden? Gebruik je hiervoor literatuurstudie, interviews met belanghebbenden (bv.~voor requirements-analyse), experimenten, simulaties, vergelijkende studie, risico-analyse, PoC, \ldots?

Valt je onderwerp onder één van de typische soorten bachelorproeven die besproken zijn in de lessen Research Methods (bv.\ vergelijkende studie of risico-analyse)? Zorg er dan ook voor dat we duidelijk de verschillende stappen terug vinden die we verwachten in dit soort onderzoek!

Vermijd onderzoekstechnieken die geen objectieve, meetbare resultaten kunnen opleveren. Enquêtes, bijvoorbeeld, zijn voor een bachelorproef informatica meestal \textbf{niet geschikt}. De antwoorden zijn eerder meningen dan feiten en in de praktijk blijkt het ook bijzonder moeilijk om voldoende respondenten te vinden. Studenten die een enquête willen voeren, hebben meestal ook geen goede definitie van de populatie, waardoor ook niet kan aangetoond worden dat eventuele resultaten representatief zijn.

Uit dit onderdeel moet duidelijk naar voor komen dat je bachelorproef ook technisch voldoen\-de diepgang zal bevatten. Het zou niet kloppen als een bachelorproef informatica ook door bv.\ een student marketing zou kunnen uitgevoerd worden.

Je beschrijft ook al welke tools (hardware, software, diensten, \ldots) je denkt hiervoor te gebruiken of te ontwikkelen.

Probeer ook een tijdschatting te maken. Hoe lang zal je met elke fase van je onderzoek bezig zijn en wat zijn de concrete \emph{deliverables} in elke fase?

%---------- Verwachte resultaten ----------------------------------------------
\section{Verwacht resultaat, conclusie}%
\label{sec:verwachte_resultaten}

Hier beschrijf je welke resultaten je verwacht. Als je metingen en simulaties uitvoert, kan je hier al mock-ups maken van de grafieken samen met de verwachte conclusies. Benoem zeker al je assen en de onderdelen van de grafiek die je gaat gebruiken. Dit zorgt ervoor dat je concreet weet welk soort data je moet verzamelen en hoe je die moet meten.

Wat heeft de doelgroep van je onderzoek aan het resultaat? Op welke manier zorgt jouw bachelorproef voor een meerwaarde?

Hier beschrijf je wat je verwacht uit je onderzoek, met de motivatie waarom. Het is \textbf{niet} erg indien uit je onderzoek andere resultaten en conclusies vloeien dan dat je hier beschrijft: het is dan juist interessant om te onderzoeken waarom jouw hypothesen niet overeenkomen met de resultaten.

